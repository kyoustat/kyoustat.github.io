\documentclass[fontsize=12pt]{article}
\usepackage[utf8]{inputenc}
\usepackage[english]{babel}

\setlength{\parindent}{3em}
%\setlength{\parskip}{1em}
\renewcommand{\baselinestretch}{1.5}

% AMS Math

\usepackage[a4paper, total={7in, 9in}]{geometry}
\usepackage{amsmath}
\usepackage{amsfonts}
\usepackage{amssymb}
\usepackage{natbib}
\usepackage{graphicx}
\usepackage{url}
\usepackage{hyperref}
\usepackage{xcolor}
\usepackage{varwidth}
\usepackage{mdframed}

\newcommand{\calM}{\mathcal{M}}

\title{Rodrigues' formula for the Legendre polynomials}
\author{
	Kisung You\\
	\texttt{kisungyou@outlook.com}
}

\date{\today}



\begin{document}

\maketitle

\section{Introduction}

Legendre polynomials $P_n (x)$ are solutions of Legendre's differential equation
\begin{equation}\label{eq:legendre}
(1-x^2) y'' - 2x y' + n(n+1) y = 0\quad\text{for  } n \in \mathbb{N}\cup \{0\}
\end{equation}
and one explicit, compact expression for the polynomials is by Rodrigues' formula
\begin{equation}\label{eq:rodrigues}
P_n (x) = \frac{1}{2^n n!} \frac{d^n}{dx^n} (x^2-1)^n.
\end{equation}
This means that when $P_n (x)$ is plugged in the position of $y$ for Equation \eqref{eq:legendre}, it must satisfy the equality to 0. In this note, we show indeed the expression \eqref{eq:rodrigues} works, after a bit of tedious arithmetics. 

\section{Main}
I will proceed in two steps. Let $f_n (x) = (x^2 - 1)^n$ then we first show that the $n$-th derivative of $f_n (x)$ is a solution of Legendre equation. Then, we find a proper scaling factor of $1/ 2^n n!$ to recover $P_n (x)$ in line with a common constraint that $P_n (x) = 1$ for all $n$ when $x=1$. For notational simplicity, we denote $g^{(n)}$ for the $n$-th derivative of a function $g(x)$, i.e,
\begin{equation*}
g^{(n)} = \frac{d^n}{dx^n} g(x).
\end{equation*}

Before proceeding, we need (generalized) Leibniz's rule. Suppose we have $n$-times differentiable functions $f(x)$ and $g(x)$, then 
\begin{equation}\label{eq:leibniz} 
\frac{d^n}{dx^n} f(x) g(x) = \sum_{k=0}^n
\begin{pmatrix}
n \\ k
\end{pmatrix}
f^{(n-k)} (x) g^{(k)} (x) = \sum_{k=0}^n \frac{n!}{k! (n-k)!} f^{(n-k)} (x) g^{(k)} (x) 
\end{equation}
where the choice of $f$ and $g$ can help in reducing the number of terms when there exists a polynomial term. For example, when $g(x) = x^2$, $g^{(k)} = 0$ for all $k \geq 3$. 

\subsection*{Part 1. $f_n^{(n)} (x)$ is one solution.}
Our goal here is to show that $f_n^{(n)}(x)$ is a solution for Equation \eqref{eq:legendre}. As a first step, let's take derivative on $f_n (x)$,
\begin{align*}
\frac{d}{dx} f_n (x) &= 2 n (x^2  - 1)^{n-1} x \\
&= 2nx (x^2 - 1)^{n-1} \\
\intertext{and multiply $(x^2 - 1)$ on both sides}
(x^2 - 1) \frac{d}{dx} f_n (x)  &= 2nx (x^2 - 1)^n.
\end{align*}
Now, differentiate both sides $(n+1)$ times, which leads to 
\begin{align*}
\frac{d^{n+1}}{dx^{n+1}} \left[ \frac{d}{dx} f_n (x) \right]  (x^2 -1) 
&= \sum_{k=0}^{n+1}
\begin{pmatrix}
n+1 \\ k
\end{pmatrix}
\left( \frac{d}{dx} f_n (x) \right)^{(n+1-k)} (x^2-1)^{(k)} \\
&= 
\begin{pmatrix}
n+1 \\ 0
\end{pmatrix}
f_n^{(n+2)} (x) (x^2-1) + 
\begin{pmatrix}
n+1 \\ 1
\end{pmatrix}
2x f_n^{(n+1)}  (x) +
\begin{pmatrix}
n+1 \\ 2
\end{pmatrix}
f_n ^{(n)} (x) \cdot 2 \\
&= (x^2-1) f_n^{(n+2)} (x) + 2(n+1)x f_n^{(n+1)} (x) + 
n(n+1) f_n^{(n)} (x) \\
\intertext{for the left-hand side, and}
\frac{d^{n+1}}{dx^{n+1}} f_n(x) 2nx &=
\begin{pmatrix}
n+1 \\ 0
\end{pmatrix} 
f_n^{(n+1)} (x) 2nx + 
\begin{pmatrix}
n+1 \\ 1
\end{pmatrix}
f_n^{(n)} (x) 2n \\
&= 2nx f_n^{(n+1)} (x) + 2n(n+1)f_n^{(n)}(x). 
\end{align*}
Therefore, we have following arrangement,

\begin{equation*}
	\begin{gathered}
	(x^2 - 1) f_n^{(n+2)} (x) + 2x(n+1)f_n^{(n+1)} (x) + n(n+1)f_n^{(n)} (x) = 
	2nx f_n^{(n+1)} (x) + 2n(n+1) f_n^{(n)} (x) \\
	(x^2-1) f_n^{(n+2)} (x) + 2x f_n^{(n+1)}(x) - n(n+1)f_n^{(n)} (x) = 0 \\
	(1- x^2) f_n^{(n+2)} (x) - 2x f_n^{(n+1)}(x) + n(n+1)f_n^{(n)} (x) = 0
	\end{gathered}
\end{equation*}

where the last line is in the form of Equation \eqref{eq:legendre} so that we have $f_n^{(n)} (x) $ as a solution. 


\subsection*{Part 2. find a scaling factor.}

Even though $f_n^{(n)} (x)$ as a solution, we have a requirement for the standard Legendre polynomial that $P_n (x)=1$ for $x=1$. Let us take a closer look at $f_n^{(n)}(x)$ when evaluated at $x=1$.
\begin{align*}
f_n^{(n)} (x) &= \frac{d^n}{dx^n} (x^2-1)^n \\
&= \frac{d^n}{dx^n} (x+1)^n (x-1)^n \\
\intertext{and by Leibniz's rule, we have}
&= \sum_{k=0}^n \begin{pmatrix}
n \\ k
\end{pmatrix}
\left( (x+1)^n \right)^{(k)} \left( (x-1)^n \right) ^{(n-k)} \\
&= \sum_{k=0}^n \begin{pmatrix}
n \\ k
\end{pmatrix}
\frac{n!}{(n-k)!} (x+1)^{n-k} \frac{n!}{k!} (x-1)^k \qquad (*) \\
&= n! \sum_{k=0}^n \begin{pmatrix}
n \\ k
\end{pmatrix} \frac{n!}{(n-k)! k!} (x+1)^{n-k} (x-1)^k \\
&= n! \sum_{k=0}^n \begin{pmatrix}
n \\ k
\end{pmatrix}^2 (x+1)^{n-k} (x-1)^k .
\end{align*}
Since we want to evaluate $f_n^{(n)} (x)$ at $x=1$, the last line of equations above tells us that all the terms but $k=0$ become zero, 
\begin{equation*}
f_n^{(n)} (x=1) = n! \begin{pmatrix}
n \\ 0
\end{pmatrix}^2 2^{n-0} = n! 2^n
\end{equation*}
which finally leads to define $P_n (x)$ as
\begin{equation*}
P_n (x) = \frac{1}{n! 2^n}f_n^{(n)}(x)  = \frac{1}{n! 2^n} \frac{d^n}{dx^n} (x^2-1)^n
\end{equation*}
to fulfill the condition of $P_n (x) =1$ for $x=1$. 


%\bibliographystyle{plain}
%\bibliography{references}
\end{document}
